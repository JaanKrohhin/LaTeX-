\section*{Mõisted}
\addcontentsline{toc}{section}{Mõisted}
\begin{itemize}
    \item \textbf{OOP} - Objektorienteeritud programmeerimine - see on programmeerimismudel, mis korraldab tarkvara disaini andmete või objektide, mitte funktsioonide ja loogika ümber. Objekti saab määratleda kui andmevälja, millel on ainulaadsed atribuudid ja käitumine.
    \item \textbf{Java} - laialdaselt kasutatav objektorienteeritud programmeerimiskeel ja tarkvaraplatvorm, mis töötab miljardites seadmetes, sealhulgas sülearvutites, mobiilseadmetes, mängukonsoolides, meditsiiniseadmetes ja paljudes teistes.
    \item \textbf{JFC} - Java Founding Classes - funktsioonide rühm graafiliste kasutajaliideste (GUI) loomiseks ning rikkaliku graafika funktsionaalsuse ja interaktiivsuse lisamiseks Java rakendustele.
    \item \textbf{JavaFx} - graafika- ja meediumipakettide komplekt, mis võimaldab arendajatel kujundada, luua, testida, siluda ja juurutada rikkalikke klientrakendusi, mis töötavad järjekindlalt erinevatel platvormidel.
    \item \textbf{Swing} - kerge Java graafilise kasutajaliidese (GUI) vidinate tööriistakomplekt, mis sisaldab rikkalikku komplekti vidinaid.
    \item \textbf{CSS} - Cascading Style Sheets - describes how elements should be rendered on screen, on paper, in speech, or on other media.
    \item \textbf{Scene Builder} - visuaalne paigutuskeskkond, mis võimaldab teil kiiresti kujundada JavaFX-i rakendustele kasutajaliideseid (UI) ilma koodi kirjutamata.
    \item \textbf{FFmpeg} - avatud lähtekoodiga teek/tööriist, mis suudab mis tahes videovormingut üksteisega dekodeerida ja kodeerida.
    \item \textbf{\LaTeX }- tarkvarasüsteem dokumentide koostamiseks. Kirjutamisel kasutab kirjutaja lihtteksti, mitte vormindatud teksti, mida leidub sellistes programmides nagu Microsoft Word, LibreOffice Writer ja Apple Pages.
\end{itemize}
\newpage
