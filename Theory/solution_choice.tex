\subsection{Probleemi lahendusvariandid}
Kui tuli valida, mida projekti jaoks kasutada, oli kaks valikut: JavaFX ja Swing. Mõlemat kasutatakse Javas töölauarakenduste arendamiseks, kuid tuleb arvestada mõlemal raamistikul on oma eelised ja puudused ning valik sõltub lõppkokkuvõttes projekti konkreetsetest nõuetest.

Swing on vana ja küps raamistik, mis on olnud olemas juba pikka aega ja on praegu osa JFC'ist. See on hästi dokumenteeritud ja sellel on suur arendajate kogukond. See tähendab, et arenduse käigus tekkida võivate probleemide lahendamiseks on palju ressursse.

Teisest küljest on JavaFX kaasaegne raamistik, mis loodi tänapäevaseid kasutajaliidese põhimõtteid silmas pidades. Sellel on Swingiga võrreldes moodsam välimus ja tunnetus ning see toetab selliseid funktsioone nagu 3D-graafika, animatsioonid ja multimeedium. JavaFX-iga on kaasas ka mitmed paigutushaldurid ja kasutajaliidese juhtelemendid, mis muudavad kasutajaliidese arendamise Swingiga võrreldes kiiremaks.

Meie jaoks oli see tõesti raske valik, kuid lõpuks otsustasime JavaFxi kasuks selle kaasaegse olemuse ja selle tõttu, et see pidi asendama Swingi koha JFC-s
