\subsection{Probleemi sõnastamine}
\RaggedRight
\justifying
Tänapäeval on videotöötlusest saanud paljude tööstusharude, sealhulgas meelelahutuse, turunduse ja hariduse lahutamatu osa. Seoses kasvava nõudlusega kvaliteetse sisu järele on videotöötlustarkvara arenenud, et pakkuda nii professionaalidele kui ka entusiastidele erinevaid funktsioone ja tööriistu. Kuna need tarkvaraprogrammid on aga muutunud keerukamaks ja funktsioonirikkamaks, on nende kasutamine muutunud ka keerulisemaks. Erinevates menüüdes, nuppudes ja seadetes navigeerimise õppimine võib algajatele olla üle jõu käiv, põhjustades frustratsiooni ja järsu õppimiskõvera. Isegi kogenud kasutajad võivad veeta tunde konkreetse efekti saavutamiseks või redigeerimiseks, mis toob kaasa aja ja ressursside raiskamise. Selle tulemusena kasvab vajadus videotöötlustarkvara järele, mis oleks intuitiivsem ja kasutajasõbralikum, võimaldades kasutajatel keskenduda oma loovusele, mitte kulutada tunde tarkvara õppimisele.
\subsection{Taotluse nõuded}
Ülesande põhiolemus - luua töölauarakendus, kus kasutajad saavad kiiresti oma videoid redigeerida. Rakendusel peab olema kasutajasõbralik kasutajaliides videotöötluse algajatele. Rakendus toetab levinud videovorminguid, nagu MP4 ja AVI. Kasutajad saavad videoid kärpida ja kärbitud videoid eksportida varem kirjeldatud tavalistes videovormingutes.